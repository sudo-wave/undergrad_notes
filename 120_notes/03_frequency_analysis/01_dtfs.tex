\section{Discrete Time Fourier Series}
\begin{itemize}
	\item Start by representing signals $x(n)$ as vectors, e.g.,
	      \[
		      \bm{x} =
		      \begin{pmatrix}
			      x(0) & x(1) & x(2) & \ldots & x(p-1)
		      \end{pmatrix}^T
	      \]
	      where $p$ is the period of $x(n)$.
	\item Consider the case when $p=2$, then $\bm{x} =
		      \begin{pmatrix}
			      x(0) & x(1)
		      \end{pmatrix}^T$.
	      But, we can further decompose $\bm{x}$ into a linear combination
	      of weighted basis vectors. The basis vectors for when $p=2$ are
	      \[
		      \bm{\varphi}_{0} =
		      \begin{pmatrix}
			      1 \\
			      0
		      \end{pmatrix}
		      \quad \text{and} \quad
		      \bm{\varphi}_1 =
		      \begin{pmatrix}
			      0 \\
			      1
		      \end{pmatrix}
		      .\]
	      It follows that, $\bm{x} = \alpha_0\ \bm{\varphi_0} + \alpha_1\ \bm{\varphi_1}$
	      for some $\alpha_0, \alpha_1 \in \mathbb{Z}$.
	\item In Fourier analysis, instead of using the standard basis to represent
	      the signal $x(n)$, we use another basis (i.e., change of basis)
	      with respect to the Fourier basis vectors
	      $\bm{\psi}_0,\ \bm{\psi}_1,\ \ldots,\ \bm{\psi}_{p-1}$.
	\item Define $\psi_0(n) = e^{i0n}$ and $\psi_1(n) = e^{i\pi n}$, then
	      a $2$-periodic signal can be written as
	      \[
		      x(n) = X_0 \psi_0(n) + X_1 \psi_1(n) = X_0\ e^{i0n} + X_1\ e^{i\pi n}
	      \]
	      with respect to the Fourier coefficients $X_0, X_1$.
	\item \textbf{Note:} Only need $2$ frequencies to characterize $x(n)$,
	      both of which are integer multiples of $\pi$.
	\item More generally, if a signal is $p$-periodic, the only contributing
	      frequencies are
	      \[
		      \{0,\ \omega_0,\ 2\omega_0,\ \ldots,\ (p-1)\ \omega_0\}
		      .\]
	\item As a consequence, if
	      $x(n+p) = x(n),\ \forall\ n \in \mathbb{Z},\ \exists\ p \in \{1,2,\ldots\}$,
	      then
	      \begin{equation}
		      x(n) = X_0e^{i 0\omega_0 n} + X_1e^{i 1\omega_0 n} + \ldots + X_{p-1}e^{i(p-1)\omega_0 n}
		      .\end{equation}
	\item  \textbf{Note:} Only need $(p-1)$ Fourier coefficients since,
	      $e^{ip\omega_0n} = e^{ip\frac{2\pi}{p}n} = e^{i2\pi n} = 1$ for all $n \in \mathbb{Z}$.
	\item Going back to the case when $p=2$, we have that
	      \[
		      \bm{x} = X_0\bm{\psi}_0 + X_1\bm{\psi}_1
		      .\]
	      In order to change bases, from the standard basis of $x(n)$ to
	      the Fourier basis, we project $\bm{x}$ onto the Fourier basis
	      vectors in order to find the Fourier coefficients.
	      Projecting $\bm{x}$ onto $\bm{\psi}_0$ yields
	      \begin{align*}
		      \bm{x} \cdot \bm{\psi}_0 & = (X_0\bm{\psi}_0 + X_1\bm{\psi}_1) \cdot \bm{\psi}_0                     \\
		                               & = X_0(\bm{\psi}_0 \cdot \bm{\psi}_0) + X_1(\bm{\psi}_1 \cdot \bm{\psi}_0) \\
		                               & = X_0(\bm{\psi}_0 \cdot \bm{\psi}_0)
		      .\end{align*}
	      Therefore,
	      \[
		      X_0 = \frac{\bm{x} \cdot \bm{\psi}_0}{\bm{\psi}_0 \cdot \bm{\psi}_0} \qquad
		      \text{and} \qquad
		      X_1 = \frac{\bm{x} \cdot \bm{\psi}_1}{\bm{\psi}_1 \cdot \bm{\psi}_1}
		      .\]
	\item \textbf{Note:} Our signals are complex exponentials
	      (i.e., need not be only real-valued) so the dot product does not
	      generalize to complex vectors. Replace the dot product with the
	      inner product to bring geometry back into the signal space.
	\item Define our inner product as
	      \[
		      \langle \bm{f}, \bm{g} \rangle = f^Tg^* = \sum_{k=0}^{p-1} f_{k}g^*_{k}
		      .\]
	\item We do this because we have knowledge about vector spaces and
	      their properties, which allows us to incorporate our vector
	      space with an inner product to develop the notion of \textbf{norms}
	      and \textbf{orthogonality}.
	\item Recall that an inner product over a vector space $\mathbb{E}$ over
	      $\mathbb{C}$ or $\mathbb{R}$ is defined as
	      $\langle \cdot, \cdot \rangle : \mathbb{E} \times \mathbb{E} \longrightarrow \mathbb{C}$
	      with the following properties:
	      \begin{enumerate}
		      \item $\langle \bm{x} + \bm{y}, \bm{z} \rangle = \langle \bm{x}, \bm{z} \rangle + \langle \bm{y}, \bm{z} \rangle$
		      \item $\langle \alpha\bm{x}, \bm{y} \rangle = \alpha\langle \bm{x},\bm{y} \rangle$;
		            $\langle \bm{x}, \alpha\bm{y} \rangle = \alpha^*\langle \bm{x},\bm{y} \rangle$
		      \item $\langle \bm{x},\bm{y} \rangle^* = \langle \bm{y}, \bm{x} \rangle$
		      \item $\langle \bm{x},\bm{x} \rangle \ge 0 \text{ and } \langle \bm{x},\bm{x} \rangle = 0 \iff \bm{x} = \bm{0}$
	      \end{enumerate}
	\item The \textbf{norm} of a vector is defined as
	      $\left\lVert \bm{x} \right\lVert = \sqrt{\langle \bm{x},\bm{x} \rangle}$,
	      with the properties:
	      \begin{enumerate}
		      \item $\left\lVert \bm{x} \right\rVert \ge 0$
		      \item $\left\lVert \alpha\bm{x} \right\rVert = \left| \alpha \right| \left\lVert \bm{x} \right\rVert$
		      \item $\left\lVert \bm{x} + \bm{y} \right\rVert \le \left\lVert \bm{x} \right\rVert + \left\lVert \bm{y} \right\rVert$
	      \end{enumerate}
	\item Two vectors are \textbf{orthogonal} to each other if
	      $\langle \bm{x},\bm{y} \rangle = 0$.
	\item With all this in mind we can formally define an inner product over
	      a vetor space of signals ($p$-periodic signals):
	      \[
		      \forall\ n \in \mathbb{Z},\ f(n+p) = f(n) \text{ and } g(n+p) = g(n)
		      .\]
\end{itemize}
