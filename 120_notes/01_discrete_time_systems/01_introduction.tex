\section{Introduction}
\begin{definition}[Signal]~

	A signal is a function. That is, for some signal $x$ we take as
	input an element from $\mathbb{R}$ or $\mathbb{Z}$ and output either a
	$\mathbb{R}$ or $\mathbb{C}$ number.
\end{definition}
\begin{itemize}
	\item Put simply, signals follow the funcition rule that tells us what
	      mapping the function in question should follow.
	\item From this, we abide by the convention that whenever the domain
	      of our signal is $\mathbb{Z}$ we are working in discrete time.
	      Conversely, if the domain is $\mathbb{R}$ we will be in
	      continuous time.
\end{itemize}
\begin{definition}[Kronecker Delta a.k.a. Discrete Time Impulse]
	\[
		\delta(n) = \begin{cases}
			1, & \text{ if }n=0 \\
			1, & \text{ if }n<0
		\end{cases}
		.\]
\end{definition}
\begin{definition}[Discrete Time Unit Step]
	\[
		u(n) = \begin{cases}
			1, & \text{ if }n\ge0 \\
			0, & \text{ if }n<0
		\end{cases}
		.\]
\end{definition}
\begin{definition}[Continuous Time Unit Step]
	\[
		u(t) = \begin{cases}
			1, & \text{ if }t\ge 0 \\
			0, & \text{ if }t<0
		\end{cases}
		.\]
\end{definition}
\begin{theorem}
	Any discrete time signal can be decomposed into a linear combination of
	shifted impulses. It follows that, the unit step function can be
	reinterpreted as
	\[
		u(n) = \sum_{k=0}^{\infty} \delta(n-k)
		.\]
\end{theorem}
\begin{theorem}
	Immediately from above, we say that any discrete time signal can be
	expressed as a linear combination of shifted impulses.
	\[
		u(n) = \sum_{k=0}^{\infty} \delta(n-k) = \sum_{m=-\infty}^{n} \delta(m)
		.\]
\end{theorem}
\begin{itemize}
	\item In discrete time, the unit step is the cumulative sum of Kronecker
	      deltas. A similar concept to that of the integral.
	\item We can represent $\delta(n)$ in terms of shifted $u(n)$'s as
	      \[
		      \delta(n) = u(n) - u(n-1)
		      .\]
	      Note that
	      \[
		      \delta(n) = \frac{u(n) - u(n-1)}{n-(n-1)} = u(n) - u(n-1)
		      .\]
	      It suffices to say that this is similar to the
	      derivative.
	\item Therefore, in discrete time $\delta(n)$ is known as the
	      discrete time derivative of the impulse signal
	      and $u(n)$ is it's integral.
\end{itemize}
