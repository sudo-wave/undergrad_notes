\documentclass[11pt]{report}

%% ------------------- FONTS/ENCODING ---------------------- %%
\usepackage[utf8]{inputenc}
\usepackage[T1]{fontenc}

%% ------------------- MATH ---------------------- %%
\usepackage[leqno]{mathtools}
\usepackage{amsfonts,amsthm,amssymb,bm,tikz,cancel}
\usetikzlibrary{automata,
trees,positioning,decorations.pathreplacing,arrows.meta}
\usepackage[edges]{forest}

\makeatother
\usepackage{thmtools}
\usepackage[framemethod=TikZ]{mdframed}
\mdfsetup{skipabove=1em,skipbelow=0em}

\theoremstyle{definition}

\declaretheoremstyle[
    headfont=\bfseries\sffamily\color{blue!70!black},bodyfont=\normalfont,
    mdframed={
        linewidth=2pt,
        rightline=false,
        topline=false,
        bottomline=false,
        linecolor=blue,
        backgroundcolor=blue!5,
    }
]{thmbluebox}

\declaretheoremstyle[
    headfont=\bfseries\sffamily\color{green!70!black},bodyfont=\normalfont,
    mdframed={
        linewidth=2pt,
        rightline=false,
        topline=false,
        bottomline=false,
        linecolor=blue,
        backgroundcolor=green!5,
    }
]{thmgreenbox}

\declaretheoremstyle[
    headfont=\bfseries\sffamily\color{red!70!black},bodyfont=\normalfont,
    mdframed={
        linewidth=2pt,
        rightline=false,
        topline=false,
        bottomline=false,
        linecolor=red,
        backgroundcolor=red!5,
    }
]{thmredbox}

\declaretheoremstyle[
    headfont=\bfseries\sffamily\color{blue!70!black},bodyfont=\normalfont,
    mdframed={
        linewidth=2pt,
        rightline=false,
        topline=false,
        bottomline=false,
        linecolor=blue,
    }
]{thmblueline}

\declaretheoremstyle[
    headfont=\bfseries\sffamily\color{red!70!black},bodyfont=\normalfont,numbered=no,
    mdframed={
        linewidth=2pt,
        rightline=false,
        topline=false,
        bottomline=false,
        linecolor=red,
        backgroundcolor=red!2,
    },
    qed=\qedsymbol
]{thmproofbox}

\declaretheoremstyle[
    headfont=\bfseries\sffamily\color{blue!70!black},bodyfont=\normalfont,numbered=no,
    mdframed={
        linewidth=2pt,
        rightline=false,
        topline=false,
        bottomline=false,
        linecolor=blue,
        backgroundcolor=blue!1,
    },
]{thmexplainbox}

% \declaretheorem[style=thmredbox,name=Theorem]{theorem}
% \declaretheorem[style=thmproofbox,name=Proof]{replacementproof}
% \renewenvironment{proof}[1][\proofname]{\vspace{-10pt}\begin{replacementproof}}{\end{replacementproof}}

\newtheorem{theorem}{Theorem}[chapter]
\newtheorem{axiom}{Axiom}
\newtheorem{lemma}{Lemma}
\newtheorem{proposition}{Proposition}
\newtheorem{definition}{Definition}[chapter]
\newtheorem{example}{Example}
\newtheorem{corollary}{Corollary}
\newtheorem*{remark}{Note}
\renewcommand{\qedsymbol}{$\blacksquare$}

%% ------------------- PERSONAL NOTES ---------------------- %%
\usepackage{xcolor}
\definecolor{lightergray}{gray}{0.90}
\definecolor{blizzardblue}{rgb}{0.4, 0.6, 0.8}
\definecolor{notered}{rgb}{1.0,0.1,0.1}
\definecolor{thoughtblue}{rgb}{0.1,0.1,1.0}
\newcommand{\note}[1]{\color{notered} \textbf{(#1)} \color{black}}
\newcommand{\thought}[1]{\color{thoughtblue} \textbf{(#1)} \color{black}}
\newcommand{\highlight}[1]{\color{notered} #1 \color{black}}

%% ------------------- QOL ---------------------- %%
\usepackage{parskip,graphicx,lipsum,babel,tabularx,caption,subcaption}
\usepackage[shortlabels]{enumitem}
\setlist[enumerate]{left=0pt..1em}
\setlist[itemize]{left=0pt..1em}
\usepackage[most]{tcolorbox}

%% ------------------- MARGIN/SPACING ---------------------- %%
\usepackage[a4paper,margin=1in]{geometry}
\usepackage{setspace}
\onehalfspacing

%% ------------------- HEADERS ---------------------- %%
\usepackage{fancyhdr}
\pagestyle{fancy}
\fancyhf{}
\fancyhead[L]{\leftmark}
\fancyhead[R]{\thepage}

\fancypagestyle{plain}{
  \fancyhf{}
  \renewcommand{\headrulewidth}{0pt}
  \renewcommand{\footrulewidth}{0pt}
}

\renewcommand{\chaptermark}[1]{\markboth{#1}{}}
\setlength{\headheight}{15pt}

%% ------------------- CHAPTERS/SECTIONS ---------------------- %%
\usepackage{titlesec}

\titleformat{\chapter}[display]
  {\Huge\scshape}
  {Chapter \thechapter}
  {0pt}
  {\LARGE}
  [\vspace{1ex}\hrule height 1pt \vspace{2ex}]

\titlespacing*{\chapter}{0pt}{-40pt}{0pt}

\titleformat{\section}{\Large\scshape}{\thesection}{1em}{}

\titleformat{\subsection}{\large\scshape}{\thesubsection}{1em}{}

\titleformat{\subsubsection}{\normalsize\scshape}{\thesubsubsection}{1em}{}

%% ------------------- TITLE ---------------------- %%
\usepackage{titling}

\pretitle{\vspace*{3cm} \begin{center} \LARGE}
\posttitle{\end{center} \vspace{0em}}
\preauthor{\begin{center} \large}
\postauthor{\end{center}}
\predate{\begin{center} \ttfamily}
\postdate{\end{center}}

% \title{Notes on\\[1ex] \textbf{\Huge Probability \& Random Processes}}
% \author{Irvin Avalos\\[1.5ex] \normalsize irvin.l@berkeley.edu}
% \date{}

\title{\textbf{\Huge Probability \& Random Processes}}
\author{Irvin Avalos\\[1.5ex] \normalsize irvin.l@berkeley.edu}
\date{}

\begin{document}

\maketitle

\tableofcontents

\chapter{Probability Theory}
\input{01_probability_theory/01_basics.tex}
\input{01_probability_theory/02_discrete_rvs.tex}
\input{01_probability_theory/03_continuous_rvs.tex}
\input{01_probability_theory/04_order_statistics.tex}
\input{01_probability_theory/05_transforms_and_sums.tex}
\section{Concentration Inequalities and Convergences}


\chapter{Information Theory}
\section{Information Measures and Entropy}

\input{02_information_theory/02_source_coding_theorem.tex}
\input{02_information_theory/03_huffman_coding.tex}
\input{02_information_theory/04_channel_models.tex}
\input{02_information_theory/05_channel_coding_theorem.tex}

\chapter{Stochastic Processes}
\section{Discrete Time Markov Chains}

\input{03_stochastic_processes/02_poisson_processes.tex}
\input{03_stochastic_processes/03_continuous_time_mcs.tex}

\chapter{Random Graphs}
\section{Erdos-Reyni Model}
\begin{definition}[Erdos-Reyni random graph]~

	Given $n \in \mathbb{Z}^+$ and a probability $p \in [0,1]$, the (E-R)
	random graph is denoted as $\mathcal{G}(n,p)$.
\end{definition}
\begin{itemize}
	\item The (RG) $\mathcal{G}(n,p)$ is undirected on all $n$ vertices
	\item Each of the $\binom{n}{2}$ edges is formed
	      independently from each other with probability $p$
	\item E-R stated a number of result that are based on "thresholds"
	      of $p$ which are needed to show the structural properties of
	      the graph:
	      \begin{enumerate}
		      \item $p$ is a function of $n$, $p \coloneq p(n)$
		      \item $\frac{1}{n^2}$ is the threshold for the first edges
		            to appear in a (RG)
		      \item $\frac{1}{n^{3/2}}$ is the threshold for the first
		            trees to emerge
		      \item $\frac{1}{n}$ is the threshold for the first cycles
		            to appear in the graph
	      \end{enumerate}
	\item If $p_{n} = 1/n$ then the "Giant component" emerges.
	      Let $p_{n} = (1-\varepsilon)/n$ for $0 < \varepsilon \ll 1)$,
	      then each "puddle" represents a region in the graph
	      where all nodes are connected.
	      \begin{center}
		      \begin{tikzpicture}
			      % bounding box
			      \draw (0,0) rectangle (6,4);
			      % small puddles
			      \foreach \x/\y in {1/3,2/1,4/2.5,5/3.5,3/1.2}{
					      \draw[line width=1pt] (\x,\y) circle (0.4);
				      }
			      \node[below=2mm] at (3,0) {\small Subcritical: $p=\frac{1-\varepsilon}{n}$};
		      \end{tikzpicture}
	      \end{center}
	      Note that $p_{n}$ is to the left of $\varepsilon$ and that
	      the size of the largest puddles are "small",
	      $\to O(\log n)$ w.h.p.
	\item For $p_{n} = (1+\varepsilon) / n$, we go to the
	      right of $\varepsilon \to 1 + \varepsilon$.
	      \begin{center}
		      \begin{tikzpicture}
			      % bounding box
			      \draw (0,0) rectangle (6,4);

			      % giant component (irregular blob)
			      \draw[
				      line width=1pt,
				      decorate,
				      decoration={random steps,segment length=4pt,amplitude=2pt}
			      ] (1,1) .. controls (2,3) and (4,3) .. (5,1)
			      .. controls (4,1) and (3,2) .. (1,1)
			      -- cycle;

			      % isolated nodes
			      \foreach \x/\y in {0.5/3.5,5.5/3.2,5.2/0.8,0.8/0.5}{
					      \fill (\x,\y) circle (2.5pt);
				      }

			      % caption
			      \node[below=2mm] at (3,0)
			      {\small Supercritical: \(p > \frac{1+\varepsilon}{n}\)};
		      \end{tikzpicture}
	      \end{center}
	      Here we have one large "island" where it's size is
	      relatively, $O(n)$ w.h.p.
\end{itemize}

\section{Thresholds and Connectivity}
When $p_{n}$ is below $\log_{e} n \implies$ no connectivity,
$P(\text{connectivity}) \longrightarrow 0$. Otherwise, if
$p_{n}$ is above $\log_{e} n \implies$ ensures connectivity,
$P(\text{connectivity}) \longrightarrow 0$.
\begin{center}
	\begin{tikzpicture}[node distance=4cm, every node/.style={inner sep=0}]
		% disconnected
		\node (D) {
			\begin{tikzpicture}[scale=0.8]
				\draw (0,0) rectangle (5,3);
				% a few small components
				\foreach \x/\y in {1/2,1.5/2.2,2/1.8}{
						\fill (\x,\y) circle (2pt);
					}
				\draw (1,2) -- (1.5,2.2);
				\foreach \x/\y in {3.5/1,4/1.3,4.5/0.8}{
						\fill (\x,\y) circle (2pt);
					}
				\draw (3.5,1) -- (4,1.3);
			\end{tikzpicture}
		};
		\node[below=2mm of D] {\small Disconnected: $p< \frac{\log_{e} n}{n}$};

		% connected
		\node[right=of D] (C) {
			\begin{tikzpicture}[scale=0.8]
				\draw (0,0) rectangle (5,3);
				% a path through all nodes
				\foreach \i in {1,...,5}{
						\coordinate (N\i) at (\i-1,1.5);
						\fill (N\i) circle (2pt);
					}
				\foreach \i in {1,...,4}{
						\pgfmathtruncatemacro\next{\i+1}
						\draw (N\i) -- (N\next);
					}
			\end{tikzpicture}
		};
		\node[below=2mm of C] {\small connected ($p>(1+\varepsilon)\tfrac{\log n}{n}$)};
	\end{tikzpicture}
\end{center}


\chapter{Statistical Inference}
\section{Point Estimation}
\begin{itemize}
	\item \textbf{Probability vs Statistics}: There exists two perspectives
	      that surround the field of statistical inference,
	      Bayesian vs Frequentist.
	      \begin{enumerate}
		      \item From the Bayesian perspective, unknown quantities
		            are treated as RVs with known or assumed distributions and priors.
		      \item While under the Frequentist's point of view, unknowns
		            are deterministic parameters that need to be estimated.
	      \end{enumerate}
	      In essence, a person following the Bayesian perspective starts
	      with a probabilistic model and assumes its distribution and shows
	      through rigorous probabilistic analysis that measurements get
	      smaller.

	      On the other hand, a Frequentist would make hidden assumptions about
	      parameters present in the model for which we know knowing about.
	\item \textbf{Bayesian Inference:}
\end{itemize}



\end{document}
